%----------------------------------------------------------------------------------------
%	MSC SUBJECT SECTION
%----------------------------------------------------------------------------------------

\section{   \en{Master's Thesis}
            \fr{Mémoire}
            \sv{Examens Arbete}}
% English
\en{
	\cvitem{Title}{\emph{Fast Oscillation Detection using Synchrophasors Signals}}
	\cvitem{Supervisor}{Professor Luigi Vanfretti}
	\cvitem{Description}{
	The Thesis focused on the development and implementation of fast oscillation detection algorithms for real-time monitoring tools.
    The PMU measurement-based prototype application allow real-time analysis of higher frequency events (10-20~Hz).
	\begin{itemize}
		\item Developed real-time models for \textbf{Opal-RT} simulator.
		\item Configured and used several PMUs and PDCs from different vendors.
		\item Developed a prototype application in \textsc{LabVIEW}.
	\end{itemize}
	}
}

% Francais
\fr{\cvitem{Titre}{\emph{Fast Oscillation Detection using Synchrophasors Signals}}
	\cvitem{Tuteur}{Professor Luigi Vanfretti}
	\cvitem{Description}{Oscillatory events around 13~Hz have been recorded in the US by Oklahoma Gas \& Electric (OG\&E). They have been identified as coming from the interaction of two windfarms. Such a high frequency is very different from the traditional and well studied Inter-area oscillations, it is also beyond the measurement capabilities of most of the existing measurement equipments and monitoring tools.\newline{}
		This Thesis focuses on the development and implementation of algorithms for oscillation detection which can support real-time monitoring tools. It proposes a real-time monitoring tool that exploits synchronized phasor measurements from PMUs, which allow real-time analysis of higher frequency events.
		\newline{}\newline{}
		Detailed achievements:
		\begin{itemize}
			\item Developed real-time models for \textbf{Opal-RT} simulator.
			\item Configured and used several PMUs and PDCs from different vendors.
			\item Developed a monitoring application processing PMU measurements in \textsc{LabView}.
		\end{itemize}
	}
}

% Svenska

