%----------------------------------------------------------------------------------------
%	PHD SUBJECT SECTION
%----------------------------------------------------------------------------------------


\section{\en{PhD Subject}}
\en{
	\cvitem{Title}{\emph{PMU-based Model Calibration for Power Systems and Control Optimization}}
	\cvitem{Supervisor}{Professor Luigi Vanfretti}
	\cvitem{Description}{
        The network operation process aims to make a higher use of simulation tools to take preventive decisions. This requires however a strong confidence in the models used, which can only be achieved with calibration for a specific purpose. \newline
        This project focuses on the development of methods and a working prototype for near real-time model calibration for small signal stability studies.
        The project will make use of PMU measurements and parameter estimation methods to calibrate models. Once the modes from the calibrated model match the ones estimated from the measurements, the model will be considered as calibrated for this instant. \newline
        Detailed achievements:
        \begin{itemize}
            \item Developed and maintained a \textsc{Modelica} library for power system simulation.
            \item Developed and maintained a \textsc{MatLab} toolbox for \textbf{model calibration}.
            \item Developed prototype applications in \textsc{LabVIEW}.
        \end{itemize}	
	}
}

\fr{
    \cvitem{Titre}{\emph{}}
    \cvitem{Tuteur}{Professeur Luigi Vanfretti}
    \cvitem{Déscription}{
    }

}
